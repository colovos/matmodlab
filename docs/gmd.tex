\documentclass[11pt]{report}
\usepackage{ifthen}
\newcommand{\gmd}{\texttt{gmd}}
\newcommand{\tag}[1]{\texttt{<#1>}}
\newcommand{\gat}[1]{\texttt{<\textbackslash{}#1>}}
\newcommand{\element}[4][0in]{%
  \indent\hspace{#1}\tag{#2 #3}\\%
  \indent\hspace{.2in}#4\\%
  \indent\hspace{#1}\gat{#2}}
\newcommand{\attr}[4]{%
  \ifthenelse{\equal{#3}{}}{%
    \ifthenelse{\equal{#4}{}}{%
      \texttt{#1=``#2''}}{%
      \texttt{#1=``#2\{#4\}''}}}{%
    \ifthenelse{\equal{#4}{}}{%
      \texttt{#1=``#2[#3]''}}{%
      \texttt{#1=``#2[#3]\{#4\}''}}}}


\title{\gmd{} Users Guide}

\begin{document}
\maketitle

% ----------------------------------------------------------------------------- %
\chapter{Introduction}
\gmd{} is a Generalized Model Driver.

% ----------------------------------------------------------------------------- %
\chapter{User Input}
User input is via xml control files. In general, tags use CamelCase and
attributes lower case.

% ----------------------------------------------------------------------------- %
\section{GMDSpec}
\begin{verbatim}
<GMDSpec>
\end{verbatim}
%
All input files must have as their root element \tag{GMDSpec}. Recognized
subelements of \tag{GMDSpec} are
%
\begin{itemize}
  \item \tag{Physics}
  \item \tag{Permutation}
  \item \tag{Optimization}
\end{itemize}

Additionally, the following elements are read from anywhere in the input file
%
\begin{itemize}
  \item \tag{Include}
  \item \tag{Function}
  \item \tag{TerminationTime}
\end{itemize}

% ----------------------------------------------------------------------------- %
\section{Preprocessing}
Preprocessing allows specifying variables in the input inside of comment tags
for use in other parts of the input. Syntax mirrors that of \texttt{aprepro}.

\subsection{Example}
Specify the \tag{Material} parameter \texttt{K} and \tag{Path} parameter
\texttt{estar} as variables
\begin{verbatim}
<GMDSpec>
  <!-- {K = 23e9}
       {estar = -.05}
  -->
  <Physics>
    <Material model="elastic">
      <K> {K} </K>
      <G> 54e9 </G>
    </Material>
    <Path type="prdef" estar="{estar}">
         . . .
    </Path>
  </Physics>
</GMDSpec>
\end{verbatim}

% ----------------------------------------------------------------------------- %
\section{Include}
\begin{verbatim}
<Include href="str"/>
\end{verbatim}
%
Path to file to be included as if its contents were inplace in the input file

\subsection{Example}
\begin{verbatim}
<Include href="/path/to/some/file.ext"/>
\end{verbatim}

% ----------------------------------------------------------------------------- %
\section{Function}
\begin{verbatim}
<Function id="int"
          type="str{analytic expression, piecewise linear}"
          var="str[x]" href="str" cols="list">
\end{verbatim}
%
Define functions to be used elsewhere in input. \texttt{id=1} is
reserved for the constant $1.$ function.

% ----------------------------------------------------------------------------- %
\subsection{Examples}
\paragraph{Analytic expression}
%
\begin{verbatim}
<Function id="2" type="analytic expression" var="t">
  sin(t)
</Function>
\end{verbatim}

\paragraph{Piecewise linear table}
%
\begin{verbatim}
<Function id="2" type="piecewise linear">
  1 2
  2 3
  3 5
</Function>
\end{verbatim}
%
Read a piecewise linear table from an external file using columns 1 and 3
%
\begin{verbatim}
<Function id="2" type="piecewise linear" href="./file.dat" cols="1 3"/>
\end{verbatim}

\begin{verbatim}
% cat file.dat
# Column1 Column2 Column3
1 1 4
2 3 7
.
.
.
100 4.2 1.43
\end{verbatim}

% ----------------------------------------------------------------------------- %
\section{TerminationTime}
\begin{verbatim}
<TerminationTime> float </TerminationTime>
\end{verbatim}
%
Termination time for simulation.  If not specified, termination time is taken
as final time in \tag{Path}.

\subsection{Example}
\begin{verbatim}
<TerminationTime> 1.e-6 </TerminationTime>
\end{verbatim}

% ----------------------------------------------------------------------------- %
\section{Physics}
\begin{verbatim}
<Physics driver="str[solid]{solid, eos}">
\end{verbatim}
%
Define the physics of the simulation. Recognized subelements of \tag{Physics}
are
%
\begin{itemize}
  \item \tag{Path}
  \item \tag{Material}
  \item \tag{Extract}
\end{itemize}

% ----------------------------------------------------------------------------- %
\subsection{Path}
\begin{verbatim}
<Path type="str{prdef, prstate, isotherm, hugoniot}"
      format="str[default]{default, table, fcnspec}"
      nfac="int[1]" kappa="real[0]"
      tstar="real[1]" estar="real[1]" sstar="real[1]"
      amplitude="real[1]" ratfac="real[1]">
\end{verbatim}
%
Define deformation paths.

% ----------------------------------------------------------------------------- %
\subsubsection{Examples}
\paragraph{format: default}
\begin{verbatim}
<Path type="prdef" kappa="0" tstar="1" estar="-.5" amplitude="1"
      nfac="1000" ratfac="1">
  <!-- termination time, number of steps, control, Cij -->
  0 0 222222 0 0 0 0 0 0
  1 1 222222 1 0 0 0 0 0
  2 1 222222 2 0 0 0 0 0
  3 1 222222 1 0 0 0 0 0
  4 1 222222 0 0 0 0 0 0
</Path>
\end{verbatim}

\begin{verbatim}
<Path type="prdef" nfac="100">
  0 0 444 0 0 0
  1 1 444 -7490645504 -3739707392 -3739707392
  2 1 444 -14981291008 -7479414784 -7479414784
  3 1 444 -7490645504 -3739707392 -3739707392
  4 1 444 0 0 0
</Path>
\end{verbatim}

\paragraph{format: table} Read legs from table. Control type is uniform for all
legs. Specify control type as \texttt{cfmt} attribute of \tag{Path}.
Optionally, specify the time format as \texttt{tfmt} and number of steps for
each leg as \texttt{nfac}.
\begin{verbatim}
<Path type="prdef" format="table" cols="1:4" cfmt="222222" tfmt="time">
  0 0 0 0
  1 1 0 0
    ...
  n 2 0 0
</Path>
\end{verbatim}

Read the table from a file
\begin{verbatim}
<Path type="prdef" format="table" cols="1 3:8" cfmt="222222" tfmt="time">
  <include href="exmpls.tbl"/>
</Path>
\end{verbatim}

\paragraph{format: fcnspec} Create legs from functions. Functions are specified
as \texttt{function id[:scale]}.  Syntax is otherwise similar to table format.
Only a single leg can be specified.
\begin{verbatim}
<Path type="prdef" kappa="0" tstar="1" amplitude="1" format="fcnspec"
      cfmt="222" nfac="200">
  {2 * pi} 2:1.e-1 1:0 1:0
</Path>
\end{verbatim}


% ----------------------------------------------------------------------------- %
\subsection{Material}
\begin{verbatim}
<Material model="str">
\end{verbatim}
%
Specify the material model and parameters. Material parameters are specified
as individual elements.

\subsubsection{Material database}
\begin{verbatim}
<Matlabel db="str[MTL_PARAM_DB_FILE]">
\end{verbatim}
%
Insert model parameters from a database

\subsubsection{Examples}
\begin{verbatim}
<Material model="elastic">
  <G>  54E+09 </G>
  <K> 124E+09 </K>
</Material>
\end{verbatim}

\begin{verbatim}
<Material model="elastic">
  <Matlabel db="./materials.xml"> aluminum </Matlabel>
  <K> 124E+09 </K>
</Material>
\end{verbatim}

% ----------------------------------------------------------------------------- %
\subsection{Extract}
\begin{verbatim}
<Extract format="str[ascii]{ascii, mathematic}" step="int[1]" ffmt="str[.18f]">
\end{verbatim}
%
Extract variables from exodus output to different formats. Variables to be
extracted are specified children of the \tag{Extract} element. All components
of vector and tensor variables will be extracted if only the basename is
specified.  Time is always extracted as the first entry of the output file.

\subsubsection{Examples}
Extract all components of stress and strain
%
\begin{verbatim}
<Extract format="ascii">
  STRESS STRAIN
</Extract>
\end{verbatim}

Extract only the XX, YY, and ZZ components of stress
%
\begin{verbatim}
<Extract format="ascii">
  STRESS_XX STRESS_YY STRESS_ZZ
</Extract>
\end{verbatim}

Extract everything
\begin{verbatim}
<Extract format="ascii">
  ALL
</Extract>
\end{verbatim}

% ----------------------------------------------------------------------------- %
\section{Permutation}
\begin{verbatim}
<Permutation method="str[zip]{zip, combine}" seed="real[12]">
\end{verbatim}
%
Permutate model input parameters to investigate sensitivities. Recognized
subelements of \tag{Permutation}

\begin{itemize}
  \item \tag{Permutate}
\end{itemize}

% ----------------------------------------------------------------------------- %
\subsection{Permutate}
\begin{verbatim}
<Permutate var="str"
           values="fcn{range, list, weibull, uniform, normal, percentage}"
\end{verbatim}
%
Specify the paramaters to permutate. Variable names should occur elsewhere in
the input file in preprocessing braces.

\subsection{Example}
Permutate the \texttt{K} and \texttt{G} parameters
%
\begin{verbatim}
<Permutation method="zip" seed="12">
  <Permutate var="K" values="weibull(125.e9, 14, 3)"/>
  <Permutate var="G" values="percentage(45.e9, 10, 3)"/>
</Permutation>
\end{verbatim}

In the \tag{Material} element, the \texttt{K} and \texttt{G} parameters are
specified as
%
\begin{verbatim}
<Material model="elastic">
  <K> {K} </K>
  <G> {G} </G>
</Material>
\end{verbatim}


% ----------------------------------------------------------------------------- %
\section{Optimization}
\begin{verbatim}
<Optimization method="str[simplex]{simplex, powell, cobyla}"
              maxiter="int[25]" tolerance="real[1e-6]">
\end{verbatim}
%
Optimize specified parameters against user specified objective function.
Recognized subelements of \tag{Optimization}

\begin{itemize}
  \item \tag{Optimize}
  \item \tag{AuxiliaryFile}
  \item \tag{ObjectiveFunction}
\end{itemize}

% ----------------------------------------------------------------------------- %
\subsection{Optimize}
\begin{verbatim}
<Optimize var="str" initial_value="real" bounds="list"/>
\end{verbatim}
%
Specify the variable to be optimized, giving initial value and, optionally,
bounds.  Only the \texttt{cobyla} method accepts bounds.  Variable names should
occur elsewhere in the input file in preprocessing braces.

% ----------------------------------------------------------------------------- %
\subsection{AuxiliaryFile}
Path to any auxiliary file needed by the optimization objective function.
\begin{verbatim}
<AuxiliaryFile href="str"/>
\end{verbatim}

% ----------------------------------------------------------------------------- %
\subsection{ObjectiveFunction}
Path to a user defined executable script that returns the error to the
optimization routine.
\begin{verbatim}
<ObjectiveFunction href="str"/>
\end{verbatim}

An \texttt{ObjectiveFunction} \texttt{scriptname} is called from the command
line as
\begin{verbatim}
% ./scriptname simulation_output.exo [auxiliary_file_1 [... auxiliary_file_n]]
\end{verbatim}

\subsection{Example}
Optimize the \texttt{K} and \texttt{G} parameters
\begin{verbatim}
<Optimization method="simplex" maxiter="25" tolerance="1e-4" disp="0">
  <ObjectiveFunction href="opt-sig-v-time"/>
  <AuxiliaryFile href="opt-baseline.dat"/>
  <Optimize var="opt_k" initial_value="129.e9"/>
  <Optimize var="opt_g" initial_value="54.e9"/>
</Optimization>
\end{verbatim}

In the \tag{Material} element, the \texttt{K} and \texttt{G} parameters are
specified as
%
\begin{verbatim}
<Material model="elastic">
  <K> {opt_k} </K>
  <G> {opt_g} </G>
</Material>
\end{verbatim}

\end{document}

%%% Local Variables:
%%% mode: latex
%%% TeX-master: t
%%% End:
